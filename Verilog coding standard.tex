\RequirePackage[l2tabu, orthodox]{nag} % 检测是否使用了已经淘汰的宏包与命令
\documentclass[utf-8, 10pt, a4paper, titlepage, oneside, onecolumn, openany]{ctexart} % 编码格式,字体大小,纸张大小,创建标题页,单面(页码在同一侧),单列排版,新章的开始不分奇偶,

\usepackage{hologo} % 包含各种 Latex logo
\usepackage{metalogo} % 包含各种 Latex logo
\usepackage{amsmath} % 数学公式的支持与优化
\usepackage{siunitx} % 提供学术单位与数字标准写法

%%%
\usepackage{fontspec} % 提供修改字体的功能
\setmainfont{Times New Roman} % 英文主字体
% \setsansfont % 英文无衬线字体
\setmonofont{Source Code Pro} % 英文等宽字体
% \setCJKmainfont % 中文主字体
% \setCJKsansfont % 中文无衬线字体
% \setCJKmonofont{Source Code Pro} % 中文等宽字体
%\setmathrm{Optima} % 数学公式字体
% \setmathsf[<font options>]{<font name>} % 数学公式字体
% \setmathtt[<font options>]{<font name>} % 数学公式字体
% \setboldmathrm[<font options>]{<font name>} % 数学公式字体
%%%

\usepackage{xcolor} % 颜色定义

%%% 阴影环境设置
\usepackage{framed}
\colorlet{shadecolor}{red!20!green!20!blue!20}
%%%

%%% 代码环境设置
\usepackage{listings}
% \lstset{language = Verilog}
\lstset{
	basicstyle = \fontspec{Source Code Pro}, % 代码字体设置
	backgroundcolor = \color[RGB]{245, 245, 244}, % 背景颜色
	frame = shadowbox, % 阴影框效果
	rulesepcolor = \color{red!20!green!20!blue!20}, % 代码框边框为淡青色
	numbers = left, % 显示行号
	stepnumber = 1, % 若设置为2,则显示行号为1,3,5,即 stepnumber 为公差,默认为 1
	numberstyle = \normalsize, % 行号字体大小
	numbersep = 16pt, % 设置行号与代码的距离,默认是 5pt
	tabsize = 4, % table 宽度
	keywordstyle = \color{blue!70}\bfseries, % 代码关键字为蓝色
	% keywordstyle=[1]\color{Blue}\bfseries, % 内置函数为蓝色粗体
	% keywordstyle=[2]\color{Purple}, % 函数参数为紫色
	% keywordstyle=[3]\color{Blue}\underbar, % 用户函数为蓝色并加下划线
	stringstyle = \rmfamily\slshape\color[RGB]{128,0,0}, % 字符串为紫色
	commentstyle = \color{red!50!green!50!blue!50}, % 浅灰色的注释
	stringstyle = \ttfamily, % 代码字符串的特殊格式
	showstringspaces = false, % 不显示代码字符串中间的空格标记
	breaklines=true, % 对过长的代码自动换行
%	escapeinside = ``, % 逃逸符号,该符号中可写入中文
	xleftmargin = 4em, % 代码框左边空白宽度
	xrightmargin = 4em, % 代码框右边空白宽度
	aboveskip = 2em, % 代码框与上文的距离
	framexleftmargin = 2em % 代码框左边框与代码的距离,默认行号不在框内,2 em 令行号在框内
}
%%%

\usepackage[cache=false]{minted} % 代码高亮,使用该宏包需要开启 --shell-escape,在编译命令中修改为 xelatex.exe -synctex=1 -interaction=nonstopmode --shell-escape -8bit %.tex 增加 -8bit 选项是为了防止 tab 显示异常

\usepackage{txfonts} % 文本与数学符号库,且默认将正文英文改为 Times Roman 字体
\usepackage{upgreek} % 提供直立希腊字母,将 $\mu$ 改为 $\upmu$

%%% 交叉引用环境设置,红色无边框,此处等号两边不可加空格
\usepackage[colorlinks=true, linkcolor=red, citecolor=blue, menucolor=green, pdfborder={0,0,0}, unicode, pdfstartview=FitH, bookmarks=true, bookmarksopen=true, hyperfootnotes=true]{hyperref}

\usepackage{tabularx} % 灵活控制表格的生成
\usepackage{booktabs} % 创建没有竖线分隔的表格,即学术表格

%%% 为目录中section补上引导点
\usepackage[titles]{tocloft} 
\renewcommand\cftsecdotsep{\cftdotsep}
\renewcommand\cftsecleader{\cftdotfill{\cftsecdotsep}} 
%%%

% \usepackage[left = 1.25in,right = 1.25in,top = 1in,bottom = 1in]{geometry} % 调整页面边距
\raggedbottom % 页面顶部对齐

%%% 页眉页脚设置
\usepackage{fancyhdr}
\pagestyle{fancy} % 使用 fancy 风格
\fancyhf{} % 清空页眉页脚
\fancyhead[L]{\bfseries\rightmark} % 页眉居左
%\fancyhead[L]{\bfseries\leftmark} % 页眉居左
\fancyfoot[C]{\thepage} % 页脚居中
\renewcommand{\headrulewidth}{0.5pt} % 页眉增加下划线并设置线宽
%%%

\usepackage{makeidx} % 提供索引功能
\makeindex % 生成索引

\usepackage{microtype} % 改善单词、字母的间距,需要放到字体宏包后面

\begin{document}
	\renewcommand{\thefootnote}{\fnsymbol{footnote}} % 符号脚注

	\pdfbookmark[1]{封面}{封面} % 手动书签

	\title{\textbf{Verilog 编码规范}} % 标题页面标题
	\author{吕旭东 \thanks{~齐奉先} \footnote{~于上海天马} \footnote{~\TeX ~ \LaTeX ~ \LaTeXe ~ \XeLaTeX $^{\cite{Latex介绍}}$}} % 标题页面作者及脚注
	\date{\today} % 标题页面日期
	\maketitle % 创建标题页面

	\tableofcontents \addcontentsline{toc}{section}{目录}	% 创建带链接目录
	\clearpage % 换页

	\renewcommand{\thefootnote}{\arabic{footnote}} % 数字脚注


	\section{宗旨}
		规范、严谨、优雅。
		
		规范意为代码编写必须符合本文档的准则。当涉及到本文档未曾描述的情况时,则需以严谨的态度处理,以蹲循本文档的精神。当达成同一目的有多种方法可行时,应尽可能以最优雅的方式实现。
		
	\section{命名规范}

		\begin{enumerate}
			\item parameter 与  {\fontspec{Source Code Pro}`}define 定义的名称全部大写。 % 该方法改变局部字体
			
			\item module 名的每个单词首字母大写,不使用下划线 \footnote{~定义 module 时名称中不使用下划线,但例化 module 时的实体名加下划线再加“Un”,n 为实体序号,参见代码表~\ref{格式示例}~。},后加层次级别 Vn,n 为层次级别,顶层 module 为 0,子 module 依次递增。例如:

				\begin{table}[h]
					\caption{命名示例 0} \label{命名示例0}
				\end{table}

				\begin{shaded}
					%%% 数学公式逃逸,行号居左,行号与左框线距离,tab 4 格,去除左端 6 个 tab,加上代码框线,代码与左框线距离,左右空白宽度
					\begin{minted}[mathescape,numbers=left,numbersep=4pt,tabsize=4,gobble=6,frame=single,framesep=4pt,xleftmargin=2em,xrightmargin=2em]{Verilog}
						`define MAX 99
						parameter NUMBER 64
						module TopV0();
						module KeyControlV1();
					\end{minted}
				\end{shaded}

			\item reg、wire、output 与 input 信号名全部使用小写字母 \footnote{~唯一的例外是低电平有效的信号,需要加大写字母 N,参见代码表 ~\ref{命名示例1}~ 或代码表 ~\ref{命名示例2}~。},单词间用下划线连接。默认为标量 wire,若为 reg 变量则加后缀 r,向量加后缀 v,输入变量加后缀 i,输出变量加后缀 o,低电平有效变量下划线前加 N。例如:
			
			\clearpage

				\begin{table}[h]
					\caption{命名示例 1} \label{命名示例1}
				\end{table}

\begin{lstlisting}[language = Verilog]
`timescale 1 ns / 1 ps

module TopV0(
	input wire clk_i,
	input wire resetN_i,
	input wire[7:0] data_iv,
	output reg hsync_o,
	output reg[7:0] rdata_ov
);
\end{lstlisting}

				\begin{table}[h]
					\caption{命名示例 2} \label{命名示例2}
				\end{table}

				\begin{shaded}
					%%% 数学公式逃逸,行号居左,行号与左框线距离,tab 4 格,去除左端 6 个 tab,加上代码框线,代码与左框线距离
					\begin{minted}[mathescape,numbers=left,numbersep=4pt,tabsize=4,gobble=6,frame=single,framesep=4pt,xleftmargin=2em,xrightmargin=2em]{Verilog}
						`timescale 1 ns / 1 ps
						
						module TopV0(
							input wire clk_i,
							input wire resetN_i,
							input wire[7:0] data_iv,
							output reg hsync_o,
							output reg[7:0] rdata_ov
						);
					\end{minted}
				\end{shaded}

		\end{enumerate}

	\clearpage

	\section{代码格式}
		代码格式使用 Java 风格的缩进方式$^{\cite{Java 编程思想}}$,以 Java 中的 “\{” 等价 Verilog 中的 “begin”,Java 中的 “\}” 等价 Verilog 中的 “end”。编码格式使用 gbk \footnote{~考虑到 Quartus II 仅支持 gbk 编码,所以为避免乱码出现,所有编辑器统一使用 gbk 编码。},缩进使用 Table,且 Table 宽度为 4 个空格。运算符左右各加一个空格,逗号后加一个空格,关键字与小括号之间加一个空格, module 名与小括号间不加空格,关键字与中括号间不加空格。下述例程。
		
		\begin{table}[h]
			\caption{格式示例} \label{格式示例}
		\end{table}
		
		\begin{shaded}
			\begin{minted}[mathescape,numbers=left,numbersep=4pt,tabsize=4,gobble=4,frame=single,framesep=4pt,xleftmargin=1em,xrightmargin=0em]{Verilog}
				`timescale 1 ns / 1 ps
				`define MAX 512
				
				module TestAgingV0(
					input wire clk_i,
					input wire resetN_i,
					output reg[23:0] pic_data_ov
				);
				
					parameter ONE_STEP = 16;
					parameter TWO_STEP = 32;

					reg[15:0] count_rv;
					reg[7:0] picture_rv;
					reg enable_r;
					wire flag_w;
					wire[7:0] number_wv;
					
					initial begin
						count_rv <= 16'd0;
						picture_rv <= 8'd0;
						enable_r <= 1'd1;
					end
					
					assign flag_w = enable_r;
					assign number_wv = picture_rv;
					
					always @(posedge clk_i) begin
						if (resetN_i == 0) begin
							picture_rv <= 0;
							count_rv <= 0;
						end else begin
							count_rv <= count_rv + 1;
							if ((count_rv == ONE_STEP) begin
								picture_rv <= 1;
							end else if (count_rv == TWO_STEP) begin
								picture_rv <= 2;
							end else if (count_rv >= MAX) begin
								picture_rv <= 0;
								count_rv <= 0;
							end
						end
					end

					always @(posedge clk_i) begin
						if (enable_r == 1) begin
							case (picture_rv)
							0: begin
								pic_data_ov <= 0;
							end
							1: begin
								pic_data_ov <= 1;
							end
							2: begin
								pic_data_ov <= 2;
							end
							default: begin
								pic_data_ov <= 0;
							end
							endcase
						end else begin
							pic_data_ov <= 0;
						end
					end
					
					PowerOnV1 PwoerOnV1_U0(
						.clk_i(clk_i),
						.resetN_i(resetN_i)
					);
					
					PowerOnV1 PwoerOnV1_U1(
						.clk_i(clk_i),
						.resetN_i(resetN_i)
					);
					
					PictureControlV1_U0(
						.clk_i(clk_i),
						.resetN_i(resetN_i),
						.picture_iv(picture_rv[7:0])
					);
					
				endmodule

			\end{minted}
		\end{shaded}

	\clearpage

	\section{编程风格}
		\begin{enumerate}
			\item 定义 module 时其输入输出端口必须在 module 的声明区\footnote{~即 module 名后的小括号内。}完成,并且每个端口都必须显式声明其属于 reg 类型还是 wrie 类型。
			
			\item module 的 input 端口定义为 wire 类型,output 端口定义为 reg 类型,inout 端口\footnote{~例如 IIC 的 SDA 端口即为 inout 类型,因 IIC 通信既有读亦有写。} 定义为 wire 类型,除极特殊的应用场景外,必须如此执行。即输入信号不能寄存,而输出信号必须寄存。此举的目的在于防止出现 module 例化与互联时的端口类型不匹配错误。
			
			\item 例化 module 时,如果连接的信号是向量,则必须显式指明其位宽,参见代码表~\ref{格式示例}~。
			
			\item module 端口名与其连接的信号名应尽可能保持一致\footnote{~除表示信号类型的后缀之外。}。
			
		\end{enumerate}

	\clearpage

	\begin{thebibliography}{9}
		\bibitem{Latex介绍} Tobias Oetiker, Hubert Partl, Irene Hyna and Elisabeth Schlegl. China\TeX 论坛译. 一份不太简短的 \LaTeXe 介绍[R]. 2016.
		\bibitem{Java 编程思想} [美] Bruce Eckel. 陈昊鹏译. Java编程思想:第4版[M]. 机械工业出版社, 2007.
		\bibitem{Verilog 标准} IEEE Computer Society. IEEE Standard for Verilog$^{\textregistered}$ Hardware Description Language[R]. IEEE Std 1364\texttrademark -2005. USA 7 April 2006.
	\end{thebibliography}
	\addcontentsline{toc}{section}{参考文献} % 添加目录项

	\clearpage

%	\renewcommand{\numberline}[1]{\lotlabel~#1\hspace*{1em}} % 生成表格索引
	\listoftables % 生成表格索引

	\addcontentsline{toc}{section}{表格} % 添加目录项

\end{document}
